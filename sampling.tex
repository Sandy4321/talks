\documentclass[11pt,xcolor=svgnames]{beamer}
\usepackage{dsfont,natbib,setspace,changepage,multirow}
\mode<presentation>

% replaces beamer foot with simple page number
\setbeamertemplate{navigation symbols}{}
%\setbeamerfont{frametitle}{series=\bfseries}
\setbeamercolor{frametitle}{fg=Black}

\setbeamertemplate{footline}{
   \raisebox{5pt}{\makebox[\paperwidth]{\hfill\makebox[20pt]{\color{gray}\scriptsize\insertframenumber}}}}

\graphicspath{{/Users/mtaddy/Dropbox/inputs/}}
\usepackage{algorithm}
\usepackage{algorithmic}

% colors
\newcommand{\theme}{\color{Maroon}}
\newcommand{\bk}{\color{black}}
\newcommand{\rd}{\color{DarkRed}}
\newcommand{\fg}{\color{ForestGreen}}
\newcommand{\bl}{\color{blue}}
\newcommand{\gr}{\color{black!67}}
\newcommand{\sg}{\color{DarkSlateGray}}
\newcommand{\nv}{\color{Navy}}
\setbeamercolor{itemize item}{fg=gray}

% common math markups
\newcommand{\bs}[1]{\boldsymbol{#1}}
\newcommand{\mc}[1]{\mathcal{#1}}
\newcommand{\mr}[1]{\mathrm{#1}}
\newcommand{\bm}[1]{\mathbf{#1}}
\newcommand{\ds}[1]{\mathds{#1}}
\newcommand{\indep}{\perp\!\!\!\perp}
\def\plus{\texttt{+}}
\def\minus{\texttt{-}}

% spacing and style shorthand
\setstretch{1.1} 

\begin{document}

\begin{frame}[plain]

{\bf Sampling and Surveys}

\vskip .25cm
Always a key aspect of good sampling designs: figure out what is hard to learn and what is easy to learn.

\begin{itemize}
\item Stratification: over sample groups with high variance
\item Try hard to reach members of strata with low response rates.
\end{itemize}

\vskip .25cm
The same idea applies in Big Data!  \\The key is to use your computational effort (and full data) on the stuff that is hard to learn.

\begin{itemize}
\item Empirical Bayes type strategies.
\item e.g., fixing high levels in hierarchical models and using conditional independence to partition data.
\end{itemize}

\end{frame} 

\begin{frame}

{\bf So what is hard, and what is easy?}

\vskip .5cm
For us, hard = expensive and easy = cheap.

\vskip .1cm  When can we use cheap web/social data instead of a survey?

\vskip .5cm
You won't know until you try.  Do both, and try to predict the survey results from the cheap data sources.  

\vskip .2cm
If you can do this reliably, then the learning problem is easy/cheap.  If not, it is hard/expensive.

\vskip .2cm
Unfortunately, the answer will change over time.  You'll always need some survey data to keep the cheap predictor honest.
\end{frame}

\end{document}






























